\chapter{结论}
\markboth{正文}{正文}

本文探索了一种联合学习框架,该框架结合了深度图像超分辨率重建和单目深度估计两个任务,可以在不添加任何其他监督信息的情况下提升深度图像超分辨率重建的性能。

高分辨率彩色图像由于其具有与深度图像的结构相似性且容易获得,被广泛用于为深度图像超分辨率重建提供先验信息。但现有颜色指导的深度图像超分辨率重建算法通过额外分支提取到的指导信息,并没有很好地面向深度模态,因此可能在两种模态结构不一致的区域造成纹理复制等问题。而单目深度估计旨在将场景从光度表示映射到几何表示,换言之,在连续的训练和学习过程中,单目深度估计实现了从彩色图像到深度图像的跨模态信息转换。因而面向单目深度估计学习到的彩色图像特征更适合指导深度图像超分辨率重建。

本文的核心思想是如何设计两个子网络(即深度图像超分辨率重建子网络和单目深度估计子网络)之间的交互,由此本文提出了两个桥接器。特征编码阶段中的高频注意力桥将从单目深度估计子网络学习到的彩色高频信息传输到深度图像超分辨率重建子网络,从而可以提供更接近深度模态的颜色指导信息。遵循简单任务指导困难任务的原则,在特征解码阶段交换了两个任务的指导角色,深度图像超分辨率重建子网络通过内容引导桥为单目深度估计子网络在深度特征空间提供内容引导。全面的实验表明,本文提出的方法达到了具有竞争力的性能,尤其是在上采样因子较大的情况下。

此外,本文提出的网络结构具有高度的可移植性,可以为关联深度图像超分辨率重建任务和单目深度估计任务提供范例。在未来的工作中,可以通过替换不同的单目深度估计子网络和深度图像超分辨率重建子网络以更进一步验证本文提出的交互模式的普适性,并在提升深度图像超分率重建性能的同时加快网络的推理速度。
